%
% PREAMBLE
%

% !TEX TS-program = xelatex
% !TEX encoding = UTF-8 Unicode

% Generate text that can be copy-pasted
\XeTeXgenerateactualtext=1

\documentclass[12pt, parskip=full, letterpaper]{scrartcl}

% Toggle draft mode
\usepackage{etoolbox}
\newtoggle{draft}
% \toggletrue{draft}
\togglefalse{draft}

% Config variables
\newcommand{\indentvalue}{1.5em}

% Layout and Appearance
\usepackage{libertinus}
\usepackage{lettrine}
\usepackage[american]{babel}
\usepackage[en-US]{datetime2}
\DTMlangsetup[en-US]{ord=raise}
\parindent=\indentvalue
\iftoggle{draft}{
    % Drafts use 1 inch margins and double spacing on body text
    \usepackage[margin=1in]{geometry}
    \usepackage[nodisplayskipstretch, doublespacing]{setspace}
    \addtokomafont{titlehead}{\setstretch{1}}
    \addtokomafont{subject}{\setstretch{1}}
    \addtokomafont{title}{\setstretch{1}}
    \addtokomafont{subtitle}{\setstretch{1}}
    \addtokomafont{author}{\setstretch{1}}
    \addtokomafont{date}{\setstretch{1}}
    \addtokomafont{publishers}{\setstretch{1}}
    \setcounter{DefaultLines}{2}
}{
    \setcounter{DefaultLines}{3}
}

% Bibliography
\usepackage[style=mla, backend=biber]{biblatex}
\addbibresource{references.bib}

% Quotations
\usepackage[indentfirst=false, leftmargin=\indentvalue]{quoting}
\usepackage[threshold=1]{csquotes}
\SetBlockEnvironment{quoting}
\SetCiteCommand{\autocite}

% References
\usepackage{hyperref}
\hypersetup{
unicode=true,
pdfstartview={FitH},
colorlinks=true,
citecolor=blue,
}


% Title
\titlehead{Princeton University}
\title{Woolf, Hume, \& Sense}
\author{William Svoboda}
\subject{Ways of Knowing: Philosophy and Literature}
\date{\DTMdisplaydate{2021}{10}{13}{}}

%
% BODY 
%

\begin{document}

\maketitle

\lettrine[nindent=0pt, findent=0.25em, loversize=0.09]{T}{he} philosopher David Hume wrote in his \emph{Treatise of Human Nature} that all human perception is divided between \enquote{\textsc{Impressions} and \textsc{Ideas}} \autocite[1]{hume_78}. For Hume, the strongest experiences were limited to the direct impressions of our sensations, passions and emotions \autocite[1]{hume_78}. Virginia Woolf, author of \emph{To the Lighthouse}, uses sense imagery throughout the novel to highlight key developments. Woolf was certainly aware of philosophers like Hume, and indeed the character of Mr. Ramsay is a playful nod towards their work. However, Woolf does not constrain herself to Hume's philosophy. At several points during the novel she challenges Hume through her use of memory and metaphor. Ultimately, Woolf seems to argue that our recollections of the past can be just as vivid as our impressions of the present moment.

\emph{To the Lighthouse} is a story about people and the relationships between them. Its characters are messy and flawed, and yet Woolf takes the time to show their growth and development in spite of this. Lily Briscoe, one of the novel's central characters, has an arc that lasts all the way until the final pages. Lily is an artist, spending most of the novel intent on finishing a single painting. The time spent with this painting is the catalyst for Woolf's exploration of her character.

On page 201, for example, Lily is described as being lost in thought. In the midst of painting, she had fallen into a daydream about her relationship with Mr. Ramsay. It is only after dropping the flowers she had been holding that Lily is snapped back to the task at hand. Woolf describes her reaction to this development:

\blockcquote[][201]{woolf_81}[.]{She had let the flowers fall from her basket, Lily thought, screwing up her eyes and standing back as if to look at her picture which she was not touching, however, with all her faculties in a trance, frozen over superficially but moving underneath with extreme speed}

This text gives a clue to what Lily is actually experiencing. However, the situation is not resolved until the next paragraph. Here Woolf connects back to the task at hand—the painting—by showing a breakthrough in Lily's perspective on her work:

\blockcquote[][201]{woolf_81}[.]{She let her flowers fall from her basket, scattered and tumbled them onto the grass and, reluctantly and hesitatingly, but without question or complaint—had she not the faculty of obedience to perfection?—went too. Down fields, across valleys, white, flower-strewn—that was how she would have painted it. The hills were austere. It was rocky; it was steep. The waves sounded hoarse on the stones beneath}

Woolf paints a very Humean picture of Lily's thoughts. In this moment, her state of mind has become one with her impressions. She sees the world she wants to create in terms of movement, texture, and color. On page 161, a similar event occurs where Lily snaps back to reality after becoming lost in her thoughts. While the context is different (Lily had been thinking on her relationship with Charles Tansley), the outcome is almost exactly the same. As Lily experiences such a profound moment, Woolf's language shifts to describe sense imagery:

\blockcquote[][161]{woolf_81}[.]{Mrs. Ramsay making of the moment something permanent (as in another sphere Lily herself tried to make of the moment something permanent)—this was of the nature of a revelation. In the midst of chaos there was shape; this eternal passing and flowing (she looked at the clouds going and the leaves shaking) was struck into stability. Life stand still here, Mrs. Ramsay said}

In each example, she distills a major revelation into sensory particulars. While the surrounding text in both passages deal with relatively abstract ideas, they are ultimately related to us in terms of raw impressions. This structure aligns with Hume, who believed that all complex perceptions could be broken down into more basic experiences: 

\blockcquote[][4]{hume_78}[.]{Thus we find, that all simple ideas and impressions resemble each other; and as the complex are formed from them, we may affirm in general, that these two species of perception are exactly correspondent}

Moreover, he goes as far to say that \emph{all} ideas are really the result of simple impressions:

\blockcquote[][4]{hume_78}[.]{[W]e shall here content ourselves with establishing one general proposition, \emph{That all our simple ideas in their first appearance are deriv'd from simple impressions, which are correspondent to them, and which they exactly represent}}

Because Hume positions the simple impression as the strongest and most fundamental form of perception, it would seem that Woolf's use of sense imagery points to an agreement between the two writers. At the same time, Woolf complicates this possibility in one crucial way; the preceding daydream scenes are arguably just as vivid. Page 201, as previously explored, marks Lily's revelation about her painting. However, it and the preceding page also detail an event from her memory. During this recollection, we learn of a particular meal between Lily and the Ramsays:

\blockcquote[][200-201]{woolf_81}[.]{How Prue must have blamed herself for that earwig in the milk! How white she had gone when Mr. Ramsay threw his plate through the window! How she drooped under those long silences between them! Anyhow, her mother now would seem to be making it up to her; assuring her that everything was well; promising her that one of these days that same happiness would be hers. She had enjoyed it for less than a year, however}

What is striking here is Woolf's usage of free indirect discourse. Each starting with the word ``how,'' the first three sentences communicate Lily's thoughts through the voice of the narrator. Without an explicit quotation or expression like ``she said,'' Lily and the narrator effectively become one. It is because of this technique that the scene appears to be both immediate and mediated. Woolf is writing about the past, but it is because of free indirect discourse that these memories seem right in the moment.

Finally, the greatest challenge to Hume is Woolf's metaphorical language. Consider the third sentence \enquote{How she drooped under those long silences between them!} \autocite[201]{woolf_81}. Lily is most likely not, of course, literally hanging downward. The idea of drooping under silence is positional, and yet we understand what it means metaphorically: Lily feels awkward during the conflicts between Prue and Mr. Ramsay. For Woolf, metaphor is a technique to give immediacy during memory. Hume, on the other hand, would never agree that a memory could be just as vivid as an impression.

David Hume and Virginia Woolf wrote during different eras and in different genres. Both, however, ultimately comment on the nature of human perception. Woolf, although responding to many of Hume's ideas, does not necessarily disagree with them. Sensory experiences remain important throughout \emph{To the Lighthouse} even if they are not the only tool used. Woolf offers less of a rebuttal and more an expansion of Hume's philosophy. Hume himself recognized that ideas and impressions could at times be difficult to distinguish:

\blockcquote[][2]{hume_78}[.]{Thus in sleep, in a fever, in madness, or in any very violent emotions of soul, our ideas may approach to our impressions: As on the other hand it sometimes happens, that our impressions are so faint and low, that we cannot distinguish them from our ideas}

For as much as Woolf draws on Humean ideas, she never limits herself to them. Instead, she freely incorporates them into her own philosophy of perception. By exploring the gray area between impressions and ideas, Woolf shows that their intersection might be closer than even Hume could have predicted.

%
% POSTAMBLE
%

\newpage

\printbibliography

\end{document}
